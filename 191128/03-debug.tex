\section{Отладка}

\begin{frame}
	\tableofcontents[currentsection,currentsubsection]
\end{frame}

\begin{frame}{Как отлаживать}
	\begin{center}
		\includegraphics[scale=0.2]{apple-a-day.jpg}
		\pause

		An apple a day keeps the doctor away.

		Лучше предотвращать, чем отлаживать.
	\end{center}
\end{frame}

\begin{frame}{Причина}
	\begin{center}
		\includegraphics[scale=0.2]{race-or-bug.jpg}
	\end{center}
	\begin{itemize}
		\item Многопоточные баги обычно тесно связаны с порядком выполнения операций.
		\item Операции выполняются в разном порядке каждый запуск, под отладчиком, в разном коде.
		\item Очень сложно ловить баг <<за руку>>.
		\item Корректная работа на куче тестов не означает отсутствие багов.
	\end{itemize}
\end{frame}

\begin{frame}{Как предотвращать}
	\begin{itemize}
		\item Явно расставляйте инварианты в комментариях: что чем защищено, в каком порядке захватывать мьютексы.
		\item Нарисуйте на бумажке все возможные состояния системы и проверьте, что инварианты выполняются.
		\item Минимизируйте количество мьютексов, если нет проблем со скоростью работы.
		\item Не используйте для синхронизации ничего, кроме мьютексов (в частности, явных \t{sleep} в программе быть не должно).
	\end{itemize}
\end{frame}

\begin{frame}{Как тестировать}
	\begin{itemize}
		\item Запускайте на больших тестах, в которых потоки работают медленно и часто происходит переключение.
		\item
			Если вы под 64-битным Linux (или Clang под Windows) "--- используйте thread sanitizer.
			Он хорош в нахождении некоторых гонок данных, \textit{происходящих во время выполнения}.
		\item
			Аналогично можно использовать Valgrind.
		\item
			На Windows можно поставить виртуальную машину с Linux и запускать там.
		\item
			Задавайте вопросы!
	\end{itemize}
\end{frame}
