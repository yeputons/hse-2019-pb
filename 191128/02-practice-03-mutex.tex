\subsection{Взаимное исключение}

\begin{frame}
	\tableofcontents[currentsection,currentsubsection]
\end{frame}

\begin{frame}{Как избежать гонок}
	\begin{itemize}
		\item Можно обозначить кусок кода как \textit{критическую секцию} (critical section).
		\item Каждую критическую секцию может выполнять максимум один поток.
		\item
			Если весь доступ к общим данным обозначить как критическую секцию,
			то он станет де-факто атомарным.
		\item С каждой критической секцией ассоцириуется \textit{блокировка} (lock).
		\item При входе в секцию блокировку надо \textit{взять}/\textit{захватить} (acquire).
		\item При выходе из секции блокировку надо \textit{отпустить} (unlock/release).
		\item Другие названия блокировок: монитор (monitor), мьютекс (mutex, \textbf{mut}al \textbf{ex}clusion).
		\item Обычно реализованы на уровне ОС и все операции с ними медленные.
	\end{itemize}
\end{frame}

\begin{frame}{Мьютекс}
	\svgimg{mutex-good}
\end{frame}

\begin{frame}[fragile]{Некорректный пример}
\begin{minted}{cpp}
int data;
void* worker(void*) {
    pthread_mutex_t m;
    pthread_mutex_init(&m, NULL);
    for (int i = 0; i < N; i++) {
        pthread_mutex_lock(&m);
        data++;
        pthread_mutex_unlock(&m);
    }
    pthread_mutex_destroy(&m);
    return NULL;
}
\end{minted}
\href{https://raw.githubusercontent.com/yeputons/spring-2019-paradigms/master/190410/sources/09-two-threads-bad-mutex.cpp}{Код}.
\end{frame}

\begin{frame}[fragile]{Корректный пример}
\begin{minted}{cpp}
int data;
pthread_mutex_t m;
void* worker(void*) {
    for (int i = 0; i < N; i++) {
        pthread_mutex_lock(&m);
        data++;
        pthread_mutex_unlock(&m);
    }
    return NULL;
}
// ...
  pthread_mutex_init(&m, NULL);
// ...
  pthread_mutex_destroy(&m);
// ...
\end{minted}
\end{frame}

\begin{frame}{Упражнение}
	\begin{enumerate}
		\item Добавьте мьютекс в \href{https://raw.githubusercontent.com/yeputons/spring-2019-paradigms/master/190410/sources/08-two-threads.cpp}{двойной счётчик}.
		\item Уменьшите $N$ на несколько порядков (мьютексы сильно замедляют программу).
		\item Убедитесь, что всегда выводится $2N$.
		\item Добавьте мьютекс в \href{https://raw.githubusercontent.com/yeputons/spring-2019-paradigms/master/190410/sources/04-writeln-race.cpp}{writeln}.
	\end{enumerate}
	\href{https://raw.githubusercontent.com/yeputons/spring-2019-paradigms/master/190410/sources/10-two-threads-good-mutex.cpp}{Исправленный двойной счётчик},
	\href{https://raw.githubusercontent.com/yeputons/spring-2019-paradigms/master/190410/sources/11-writeln-mutex.cpp}{исправленный writeln}.
\end{frame}

\begin{frame}[fragile]{Блокировка}
	\begin{itemize}
		\item \t{pthread\_mutex\_lock} блокируется и ждёт, пока блокировка не станет доступна для захвата.
		\item Есть ли проблемы в следующем псевдокоде?
\begin{minted}{cpp}
void thread1() {
    m1.lock(); m2.lock();
    // ...
    m2.unlock(); m1.unlock();
}
void thread2() {
    m2.lock(); m1.lock();
    // ...
    m1.unlock(); m2.unlock();
}
\end{minted}
	\end{itemize}
\end{frame}

\begin{frame}{Взаимная блокировка}
	\begin{center}
		\includegraphics{deadlock.jpg}
	\end{center}
	\begin{itemize}
		\item
			Может случиться проблема:
			\begin{enumerate}
				\item Поток 1 захватывает \t{m1}.
				\item Поток 2 захватывает \t{m2}.
				\item Поток 1 не может захватить \t{m2}.
				\item Поток 2 не может захватить \t{m1}.
				\item Оба потока встали в \textit{deadlock} (\textit{взаимная блокировка}).
			\end{enumerate}
		\item
			Решение: всегда брать блокировки в одном и том же порядке.
			Тогда можно доказать, что deadlock такого вида не образуется.
		\item
			Если ввести линейный порядок на мьютексах не получается, у вас проблема.
	\end{itemize}
\end{frame}

\begin{frame}[fragile]{Мьютексы "--- не панацея}
\begin{minted}{cpp}
void inc() {  // Атомарна.
    m.lock(); data++; m.unlock();
}
void double_inc() {  // Не атомарно.
    inc(); inc();
}
void check() {
    m.lock();
    assert(data % 2 == 0);
    m.unlock();
}
\end{minted}
	\begin{itemize}
		\item Код выше может упасть, даже если вызывать только \t{double\_inc}.
		\item Мьютекс в простом случае лишь добавляет атомарности.
		\item Две атомарные операции подряд, как и раньше, атомарную вместе не образуют.
	\end{itemize}
\end{frame}

\begin{frame}[fragile]{Reentrant}
\begin{minted}{cpp}
void inc() {  // Атомарна.
    m.lock(); data++; m.unlock();
}
void double_inc() {  // Атомарно?
    m.lock(); inc(); inc(); m.unlock();
}
\end{minted}
	\begin{itemize}
		\item \t{double\_inc} заблокируется, так как \t{inc} попробует взять мьютекс второй раз.
		\item Есть специальный вид мьютексов, которые позволяют захватывать себя ещё раз из того же потока, называется \textit{reentrant}.
		\item Их обычно не используют "--- они сложнее в реализации.
	\end{itemize}
\end{frame}

\begin{frame}[fragile]{Решение}
	Вынести специальную функцию для внутреннего использования, которая предполагает, что блокировка уже взята:
\begin{minted}{cpp}
void inc_lock_held() { data++; }  // Приватная
void inc() {
    m.lock(); inc_lock_held(); m.unlock();
}
void double_inc() {
    m.lock();
    inc_lock_held(); inc_lock_held();
    m.unlock();
}
\end{minted}
	В ООП функция \t{inc\_lock\_held} обязательно была бы приватной.
\end{frame}

\begin{frame}[fragile]{Блокировка "--- ресурс}
	Есть ли проблема?
\begin{minted}{cpp}
void inc_special() {
    m.lock();
    if (condition()) return;
    data++;
    m.unlock();
}
\end{minted}
	\pause
	\begin{itemize}
		\item Есть: блокировка может быть не отпущена в случае досрочного \t{return}.
		\item Тогда \t{m} больше никто никогда не захватит и все следующие вызовы \t{inc\_special} не смогут начаться.
		\item Вам поможет привычка всегда освобождать взятые ресурсы (файлы, память, \textit{мьютекс}).
		\item Или специальный синтаксический сахар вроде RAII в C++, или try-with-resources/synchronized-блоков в Java.
	\end{itemize}
\end{frame}

\begin{frame}[fragile]{Синтаксический сахар}
	Идиома \textit{RAII}:
\begin{minted}{cpp}
void inc_special() {
    ScopedLock lock(m);  // Берём блокировку в конструкторе
    if (condition()) return;
    data++;
}  // Освобождаем в деструкторе.
\end{minted}

	synchronized-блоки в Java (каждый получит мьютекс):
\begin{minted}{java}
void inc_special() {
    synchronized {
        if (condition()) return;
        data++;
    }
}
\end{minted}
\end{frame}

\begin{frame}{Резюме}
	\begin{itemize}
		\item Атомарные операции в потоках могут выполняться в любом порядке, если их не синхронизировать.
		\item Вы обычно не знаете, что атомарно, а что нет.
		\item \textit{Любой} доступ к общим ресурсам должен быть \textit{защищён} (guarded) блокировкой.
		\item В частности, любая переменная, доступная из нескольких потоков, должна быть защищена \textit{ровно} одной блокировкой (почему?).
		\item Блокировки надо брать всегда в одном и том же порядке.
		\item Если пишете потокобезопасный объект, блокировку надо брать на самой <<верхней>> операции, которая должна быть атомарной.
		\item Дубовый способ переделки \textit{потоконебезопасной} структуры в безопасную: создать один мьютекс и брать его на каждую операцию со структурой.
		\item Мьютексы сильно тормозят, лучше вообще избегать общего доступа к данным (и меньше шанс набагать), привет из Rust.
	\end{itemize}
\end{frame}
