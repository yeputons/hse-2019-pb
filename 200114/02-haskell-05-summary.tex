\subsection{Резюме}

\begin{frame}
	\tableofcontents[currentsection,currentsubsection]
\end{frame}

\begin{frame}
	\begin{itemize}
		\item Haskell строго типизирован и есть полиморфные функции, всё проверяется на этапе компиляции.
		\item Циклов нет, переменных нет, всё делается рекурсивно.
		\item \t{if}'ы не нужны "--- используем pattern matching, в крайнем случае "--- guards.
		\item На Haskell тоже можно как-то писать в императивном стиле.
		\item Лучше писать в функциональном стиле, комбинируя функции высшего порядка со своими.
		\item Самая мощная функция из известных нам сейчас "--- \t{foldr}.
		\item Вычисления очень ленивы: пока не потребуется сравнить с чем-то, вычисления не будет.
		\item Из-за этого возможны бесконечные структуры и разумные конечные операции с ними.
		\item Списки односвязные, из-за этого они бывают бесконечными.
	\end{itemize}
\end{frame}
