\section{Функциональное программирование}
\subsection{Императивное программирование}

\begin{frame}
	\tableofcontents[currentsection,currentsubsection]
\end{frame}

\begin{frame}{Императивное программирование}
	Руководствуемся следующим принципом:
	\begin{exampleblock}{Определение}
		Алгоритм "--- набор инструкций, описывающих порядок действий исполнителя для достижения некоторого результата
	\end{exampleblock}
	\begin{itemize}
		\item В императивном стиле код программы описывает \textit{как} надо достигать результата.
		\item Есть постоянно изменяющееся состояние программы (например, переменные).
		\item Хорошо ложится на <<железо>>: оно действительно меняет состояние памяти.
	\end{itemize}
\end{frame}

\begin{frame}{Что такое <<действие>>?}
    Можно работать на разных уровнях абстракции:
	\begin{itemize}
		\item Команда процессора: <<возьми значение из ячейки памяти с таким номером>>
		\item Вычисление арифметического выражения
		\item Конструкции управления \t{if}/\t{for}/\t{while}
		\item Вызовы функций
	\end{itemize}
\end{frame}

\begin{frame}{Функциональное программирование}
	\begin{exampleblock}{Определение}
		Алгоритм "--- это \textit{чистая} математическая функция от нескольких аргументов.
	\end{exampleblock}
	\begin{itemize}
		\item Как решение задачи на курсе алгоритмов.
		\item Основное свойство "--- \textit{чистота} функций; результат вычисления зависит только от аргументов функции.
		\item Ни слова про состояние программы или порядок вычислений.
		\item Остальные свойства выводятся из чистоты:
			\begin{itemize}
				\item Порядок вычислений неважен.
				\item Состояние программы не может влиять на работу функций.
				\item Например, функция не может читать из глобальных переменных.
			\end{itemize}
	\end{itemize}
\end{frame}

\begin{frame}{Побочные эффекты}
	\begin{itemize}
		\item То, что происходит в функции помимо вычисления значения, называется \textit{побочным эффектом}.
		\item Результаты всех функций зависят только от аргументов $\iff$ у функций нет побочных эффектов.
		\item В <<идеально чистых>> языках запрещены не только функции с побочными эффектами, но любые побочные эффекты вообще.
		\item В частности, запрещено иметь изменяемое состояние "--- переменные.
		\item Никаких гонок данных.
		\item Непонятно, как тогда делать ввод-вывод "--- это точно побочный эффект.
		\item Идеально чистые языки бесполезны, все в той или иной степени <<загрязнены>>.
	\end{itemize}
\end{frame}

\begin{frame}[fragile]{Задача о рюкзаке, императивное решение}
\begin{minted}{cpp}
int maxSumCost = 0;
void solve(int i, int weight, int cost) {
  maxSumCost = max(maxSumCost, cost);  // Не чистая
  if (i == n) return;
  solve(i + 1, weight, cost);
  if (weight >= ws[i]) {
    solve(i + 1, weight - ws[i], cost + cs[i]);
  }
}
\end{minted}
\end{frame}

\begin{frame}[fragile]{Задача о рюкзаке, чистая функция-1}
\begin{minted}{cpp}
int solve(int i, int weight, int cost) {
  if (i == n) return cost;
  int ans = solve(i + 1, weight, cost);
  if (weight >= ws[i]) {
    ans = max(ans, solve(i + 1,
                         weight - ws[i],
                         cost + cs[i]));
  }
  return ans;
}
\end{minted}
\end{frame}

\begin{frame}[fragile]{Задача о рюкзаке, чистая функция-2}
\begin{minted}{cpp}
int solve(int i, int weight, int cost) {
  if (i == n) return cost;
  if (weight < ws[i]) return solve(i + 1, weight, cost);
  return max(
    solve(i + 1, weight, cost),
    solve(i + 1, weight - ws[i], cost + cs[i])
  );
}
\end{minted}
\end{frame}

\begin{frame}[fragile]{Задача о рюкзаке, чистая функция-3}
\begin{minted}{cpp}
int solve(int i, int weight) {
  if (i == n) return 0;
  if (weight < ws[i]) return solve(i + 1, weight);
  return max(
    solve(i + 1, weight),
    solve(i + 1, weight - ws[i]) + cs[i]
  );
}
\end{minted}
\end{frame}

\begin{frame}[fragile]{Императивная работа с массивом}
\begin{minted}{python}
lst = [5, 10, 15, 10, 20]
partialSums = []
smallPrefix = []
s = 0
for x in lst:
    if x >= 10:
        s += x
        partialSums.append(s)
    if x <= 10:
        smallPrefix.append(x)

for i in range(len(lst)):
    lst[i] *= 2
\end{minted}
\end{frame}

\begin{frame}[fragile]{Функциональная работа с массивом}
\begin{minted}{python}
import itertools
lst = [5, 10, 15, 10, 20]
partialSums = list(itertools.accumulate(
  # filter - "функция высшего порядка"
  filter(lambda x: x >= 10, lst)
))
smallPrefix = list(
  # takewhile- "функция высшего порядка"
  itertools.takewhile(
    lambda x: x <= 10, lst)
)
lst = [x * 2 for x in lst]
# lst = map(lambda x: x * 2, lst)
\end{minted}
\end{frame}
