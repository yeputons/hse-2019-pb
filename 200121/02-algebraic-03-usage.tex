\subsection{Использование типов-сумм}
\begin{frame}
	\tableofcontents[currentsection,currentsubsection]
\end{frame}

\begin{frame}[t,fragile]{Хранение URL}
	URL-адреса бывают:
	\begin{itemize}
		\item Относительные: \t{../images/facepalm.jpg}.
		\item Абсолютные, бывают:
			\begin{itemize}
				\item На том же домене: \t{sewiki/index.php}.
				\item На другом домене, причём:
					\begin{itemize}
						\item Та же схема (протокол): \t{google.com/humans.txt}
	                    \item Другая схема: \t{ftp://mirror.yandex.ru/}
                   	\end{itemize}
			\end{itemize}
	\end{itemize}
\begin{onlyenv}<1>
	Можно закодировать так\footnote{True story: раздел <<Thinking in Sum Types>> по \href{https://chadaustin.me/2015/07/sum-types/}{ссылке}}:
\begin{minted}{haskell}
data URL = URL (Maybe (Maybe (Maybe String, String))) String
URL Nothing        "../images/facepalm.jpg"
URL (Just Nothing) "sewiki/index.php"
URL (Just (Just (Nothing   , "google.com"))) "humans.txt"
URL (Just (Just (Just "ftp", "mirror.yandex.ru"))) ""
\end{minted}
Ужасно, не правда ли?
\end{onlyenv}

\begin{onlyenv}<2>
А можно так:
\begin{minted}{haskell}
data URL = Relative String
         | Absolute String
         | OtherDomain { domain :: String, path :: String }
         | FullUrl     { schema :: String,
                         domain :: String, path :: String }
\end{minted}
Мораль: иногда может помочь <<раскрыть по дистрибутивности>>.
\end{onlyenv}
\end{frame}
