% !TeX root = 191107.tex
\section{Парадигмы программирования}
\begin{frame}[t,fragile]{Императивная программа}
\begin{minted}{cpp}
int main(int argc, char* argv[]) {
    int i = 0, val = 0, flag = 0;
  next_arg: i++;
    if (i >= argc) goto parsed;
    if (!strcmp(argv[i], "--flag")) goto process_flag;
    if (!strcmp(argv[i], "--val")) goto process_val;
    goto next_arg;
  process_flag: flag = 1; goto next_arg;
  process_val: i++; assert(i < argc);
    val = atoi(argv[i]);
    goto next_arg;
  parsed:
    printf("val=%d, flag=%d\n", val, flag);
    return 0;
}
\end{minted}
\end{frame}

\begin{frame}[t,fragile]{Структурная программа}
\begin{minted}{cpp}
int main(int argc, char* argv[]) {
    int val = 0, flag = 0;
    for (int i = 1; i < argc; i++) {
      if (!strcmp(argv[i], "--flag")) {
        flag = 1;
      } else if (!strcmp(argv[i], "--val")) {
        i++;
        assert(i < argc);
        val = atoi(argv[i]);
      }
    }
    printf("val=%d, flag=%d\n", val, flag);
    return 0;
}
\end{minted}
\end{frame}

\begin{frame}[t,fragile]{Псевдоопределение}
	<<Парадигма программирования>> "--- это соглашение о том, \ldots
	\begin{itemize}
	\item \ldots как выражать алгоритм в коде
	\item \ldots как записывать алгоритм
	\item \ldots какие конструкции языка (не) использовать (\verb!goto!)
	\item \ldots как структурировать программу
	\item \ldots что является \textit{примитивом}
	\item \ldots как примитивы \textit{композировать}
	\item \ldots какие есть способы \textit{абстракции}
	\end{itemize}
\end{frame}

\begin{frame}[t,fragile]{Зачем изучать}
	\begin{itemize}
	\item Язык программирования "--- это инструмент, есть плюсы и минусы
	\item Парадигма "--- это инструмент, есть плюсы и минусы
	\item	На одном языка можно писать в разных парадигмах, иногда можно смешивать.
	\item Выбирайте инструмент под задачу
	\end{itemize}
	Бегло познакомимся с кучей инструментов для кругозора.

	На других курсах будете углубляться.
\end{frame}
