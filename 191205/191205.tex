\documentclass[utf8,xcolor=table]{beamer}

\usepackage[T2A]{fontenc}
\usepackage[utf8]{inputenc}
\usepackage[english,russian]{babel}
\usepackage{minted}
\usepackage{ulem}
\usepackage{cmap}
\usepackage{multirow}

\hypersetup{colorlinks,linkcolor=blue,urlcolor=blue}

\mode<presentation>{
	\usetheme{CambridgeUS}
}

\renewcommand{\t}[1]{\ifmmode{\mathtt{#1}}\else{\texttt{#1}}\fi}
\newcommand{\svgimg}[1]{
  \begin{center}
	\includegraphics[width=\textwidth,height=0.8\textheight,keepaspectratio]{#1.pdf}
  \end{center}
}

\title[Многопоточность-2]{Многопоточность-2}
\author{Егор Суворов}
\institute[НИУ ВШЭ]{Курс <<Основы программирования>>}
\date[05.12.2019]{Четверг, 5 декабря 2019 года}

\setlength{\arrayrulewidth}{1pt}

\begin{document}

\begin{frame}
\titlepage
\end{frame}

\begin{frame}{План занятия}
	\tableofcontents
\end{frame}

\section{Обмен сообщениями}
\subsection{Не пытайтесь повторить это дома}

\begin{frame}
	\tableofcontents[currentsection,currentsubsection]
\end{frame}

\begin{frame}[fragile]{Загадка}
	Что произойдёт при запуске \href{https://raw.githubusercontent.com/yeputons/spring-2019-paradigms/master/190410/sources/12-optimizer.cpp}{кода}?
	Предполагаем, что запись и чтение \t{int} атомарны.
\begin{minted}{cpp}
int data;
void* worker(void*) {
    for (;;) {
        data++;
    }
}
// ...
    while (data < 100);
    printf("Done\n");
// ...
\end{minted}
	\begin{itemize}
		\pause\item Race condition отсутствуют.
		\pause\item Он зависнет.
		\pause\item И никогда не выведет \t{Done}.
	\end{itemize}
\end{frame}

\begin{frame}{Разгадка}
	\begin{center}
		\includegraphics[scale=0.3]{optimizer.jpg}
	\end{center}
	Как обычно в C/C++.
\end{frame}

\begin{frame}{Подробная разгадка}
	\begin{itemize}
		\item Компилятор по умолчанию ничего про потоки не знает.
		\item Очевидно, что \t{while (data < 100);} переменную \t{data} изменить не может.
		\item Соответственно, переменная \t{data} никак измениться не может.
		\item Значит, \t{data < 100} всегда истинно, можно заменить на \t{true}.
		\item Получаем бесконечный цикл.
			\begin{center}
				\includegraphics[scale=0.3]{win-lose.jpg}
			\end{center}
	\end{itemize}
\end{frame}

\begin{frame}[fragile]{volatile}
	Изменим \href{https://raw.githubusercontent.com/yeputons/spring-2019-paradigms/master/190410/sources/13-volatile.cpp}{код}:
\begin{minted}{cpp}
volatile int data;  // Обозначили переменную volatile.
void* worker(void*) {
    for (;;) {
        data++;
    }
}
// ...
    while (data < 100);
    printf("Done\n");
// ...
\end{minted}
	\begin{itemize}
		\item \t{volatile} говорит компилятору честно сохранять/читать значение этой переменной из памяти каждый раз, когда написано.
		\item Есть ли проблемы? \pause Пока нет.
	\end{itemize}
\end{frame}

\begin{frame}[fragile]{Reordering}
	Эквивалентны ли два куска кода?
	\begin{tabular}{p{0.4\textwidth}p{0.4\textwidth}}
		\centering
\begin{minted}{cpp}
int data, finished;
// ...
data = 123;
finished = 1;
\end{minted}
&
\begin{minted}{cpp}
int data, finished;
// ...
finished = 1;
data = 123;
\end{minted}
	\end{tabular}
	\pause
	\begin{itemize}
		\item Эквивалентны. Оптимизатор тоже так считает.
		\pause\item И может переставить местами: всё равно никто не заметит.
		\pause\item А что, если в другом потоке было так?
\begin{minted}{cpp}
if (finished) {
    printf("%d\n", data);
}
\end{minted}
	\end{itemize}
\end{frame}

\begin{frame}{Иллюстрация}
	\begin{center}
		\includegraphics[scale=0.2]{race-condition-knock-knock.jpg}
	\end{center}
\end{frame}

\begin{frame}{Reordeing возвращается}
	\begin{itemize}
		\item Даже если одна переменная помечена как \t{volatile}, компилятор может изменить порядок записи/чтения.
		\item А вот если обе "--- не может. Проблема решена? \pause
		\item В процессоре тоже есть оптимизатор.
		\item Он тоже может переставлять инструкции как захочет, а \t{volatile} действует только на компилятор.
		\item Есть специальные ассемблерные инструкции (<<барьеры памяти>>), которые действуют на процессор.
		\item Не надо сразу пытаться в этом разобраться.
		\item \t{volatile} не предназначен для многопоточности, он нужен для других целей (memory-mapped I/O).
		\item В любом случае, иногда один поток может встретить состояние, которое было бы невозможно получить, исполняя инструкции последовательно.
	\end{itemize}
\end{frame}

\begin{frame}{А что же mutex?}
	\begin{itemize}
		\item В разных языках/библиотеках разные модели памяти (потом должны подробно рассказать про Java).
		\item Обычно везде считается, что в следующих случаях происходит (почти) полная синхронизация памяти между двумя потоками:
			\begin{enumerate}
				\item $A$ взял мьютекс, который $B$ недавно отпустил (возможно, его брал ещё кто-то).
				\item $A$ создал поток $B$.
				\item $A$ подождал завершения потока $B$.
			\end{enumerate}
		\item Все нужные барьеры памяти и прочее уже вшиты внутрь мьютексов и работы с потоками.
	\end{itemize}
\end{frame}

\begin{frame}[fragile]{Простой пример}
\begin{minted}{cpp}
// Thread 1
started = true;
m.lock();
finished = true;
m.unlock();
// Thread 2
m.lock();
if (finished) {
    assert(started);    // Верно
}
m.unlock();
\end{minted}
	Если убрать из второго потока мьютекс "--- ничего не знаем.
\end{frame}

\begin{frame}[fragile]{Сложный пример}
\begin{minted}{cpp}
// Thread 1
started = true;
m.lock(); data++; m.unlock();
finished1 = true;
finished2 = true;
// Thread 2
m.lock(); m.unlock();
if (finished2) {
    assert(finished1);  // Непонятно, потому что reordering
    assert(data > 0);   // Непонятно, потому что нет(?) happens-before
    assert(started);    // Непонятно, потому что нет(?) happens-before
}
\end{minted}
\end{frame}

\begin{frame}[fragile]{Резюме}
	\begin{itemize}
		\item Если вы что-то не защитили мьютексом, можно огрести из-за reordering, даже если всё <<очевидно должно работать>>.
		\item Если всё защищено мьютексом и вы ничего не предполагаете о происходящем за пределами критических секций "--- не о чем беспокоиться.
		\item Ничего сложнее <<взяли один глобальный мьютекс перед операцией, отпустили в конце>> обычно не требуется (в том числе в дз).
		\item Любой сколько-нибудь более сложный контроль требует понимания модели памяти.
		\item
			Все проблемы "--- от общих ресурсов (переменные, файлы, экран).
			Поэтому стараются минимизировать их количество.
		\item Нет общих ресурсов "--- нет проблем.
	\end{itemize}
\end{frame}

\subsection{Простая реализация}

\begin{frame}{Зачем}
	Довольно часто потоки не совсем независимы, а хотят взаимодействовать между собой.

	Классическая задача:
	\begin{itemize}
		\item Есть очередь задач.
		\item Один поток генерирует данные (producer) и добавляет их в очередь.
		\item Второй поток должен брать добавленные данные по очереди (consumer) и что-то с ними делать.
	\end{itemize}
	Например:
	\begin{itemize}
		\item Первый поток ждёт ввода с клавиатуры и кладёт считанные данные в буфер.
		\item Второй поток выполняет введённые команды (которые могут занять долгое время).
		\item Мы хотим уметь вводить команды, даже если предыдущая ещё выполняется.
	\end{itemize}
\end{frame}

\begin{frame}{Producer-Consumer}
	\svgimg{prod-cons}
\end{frame}

\begin{frame}[fragile]{Потокобезопасная очередь}
\begin{minted}{cpp}
class ThreadsafeQueue {
    ThreadsafeQueue() { pthread_mutex_init(&m, NULL); }
    ~ThreadsafeQueue() { pthread_mutex_destroy(&m); }
    void push(int x) {
        pthread_mutex_lock(&m);
        q.push(x);
        pthread_mutex_unlock(&m);
    }
    int pop() { ... }
    bool empty() { ... }
private:
    pthread_mutex_t m;
    queue<int> q;
};
\end{minted}
\end{frame}

\begin{frame}[fragile]{Первая попытка}
	Producer:
\begin{minted}{cpp}
while (true) {
    int data = read_int();
    q.push(data);
}
\end{minted}
	Consumer:
	\pause
\begin{minted}{cpp}
while (true) {
    while (q.empty()) {
    }
    process_data(q.pop());
}
\end{minted}
\end{frame}

\begin{frame}{Проблемы}
	\begin{itemize}
		\item Consumer активно ждёт событие от первого и тратит процессорное время.
		\item Даже если ничего не происходит, программа потребляет 100\% CPU.
		\item Consumer постоянно берёт и отпускает mutex, мешая producer'у.
		\item
			А если добавить задержку в consumer (проверять только каждые $X$ мс),
			то сильно увеличится задержка в обработке.
	\end{itemize}
\end{frame}

\begin{frame}{Есть гонка}
	\svgimg{queue-pop-race}
	Без новых \textit{примитивов синхронизации} не обойтись.
\end{frame}

\subsection{События}

\begin{frame}[fragile]{Новый примитив}
	Введём примитив \t{Event} с двумя методами:
	\begin{itemize}
		\item \t{e.wait()} "--- усыпляет поток.
		\item \t{e.notify()} "--- будит уснувший поток.
	\end{itemize}
	\begin{tabular}{p{0.45\linewidth}p{0.45\linewidth}}
		\centering
\begin{minted}{cpp}
// Producer
while (true) {
  int data = get_data();
  q.push(data);
  e.notify();
}
\end{minted}
&
\begin{minted}{cpp}
// Consumer
while (true) {
  if (!q.empty()) {
    process_data(q.pop());
  } else {
    e.wait();
  }
}
\end{minted}
	\end{tabular}
	Есть ли проблемы в коде выше?
	\pause
	Проблемы есть.
\end{frame}

\begin{frame}{Повезло}
	\svgimg{bad-event-ok}
\end{frame}

\begin{frame}{Не повезло}
	\svgimg{bad-event-bad}
\end{frame}

\begin{frame}{Ну вы поняли}
	\begin{center}
		\includegraphics[scale=0.4]{race-condition-everywhere.jpg}
	\end{center}
	Что делать?
\end{frame}

\begin{frame}{Первый подход}
	\begin{itemize}
		\item Можно сказать, что если в момент вызова \t{e.notify()} никто не спит, то будет разбужен следующий попытающийся уснуть.
		\item Другими словами, у \t{Event} теперь есть состояние: просигналили или нет.
		\item \t{e.notify()} "--- устанавливает флаг <<просигналили>> и будит все потоки.
		\item \t{e.wait()} "--- ждёт, пока флаг установят (или не ждёт, если уже установлен) и сбрасывает его.
		\item Решает задачу producer-consumer.
		\item Используются в Windows API.
	\end{itemize}
	Однако:
	\begin{itemize}
		\item Дополнительное состояние вносит сложность "--- за ним надо следить и добавлять инвариант.
		\item В pthread не входят и под Linux обычно не используются.
	\end{itemize}
\end{frame}

\begin{frame}[fragile]{Второй подход: добавим мьютексов?}
	\begin{tabular}{p{0.45\linewidth}p{0.45\linewidth}}
		\centering
\begin{minted}{cpp}
// Producer
while (true) {
  int data = get_data();
  pthread_mutex_lock(&m);
  q.push(data);
  e.notify();
  pthread_mutex_unlock(&m);
}
\end{minted}
&
\begin{minted}{cpp}
// Consumer
while (true) {
  pthread_mutex_lock(&m);
  if (!q.empty()) {
    process_data(q.pop());
  } else {
    e.wait();
  }
  pthread_mutex_unlock(&m);
}
\end{minted}
	\end{tabular}
	Теперь race condition отсутствует.
	\pause
	Зато есть deadlock: producer не может ничего писать, пока consumer спит.
\end{frame}

\subsection{Условные переменные}

\begin{frame}{Условные переменные}
	\begin{itemize}
		\item Нам нужна атомарная операция <<отпусти мьютекс и жди события>>.
		\item Такой примитив синхронизации в pthread (и вообще много где) называется \textit{условная переменная} (conditional variable).
		\item Смысл: условная переменная "--- это способ оповещать потоки о \textit{возможном} изменении некоторого \textit{условия}, защищённого мьютексом.
	\end{itemize}
\end{frame}

\begin{frame}{Событие наступило до ожидания}
	\svgimg{condvar-before-wait}
\end{frame}

\begin{frame}{Событие наступило после ожидания}
	\svgimg{condvar-after-wait}
\end{frame}

\begin{frame}{Свойства условных переменных}
	\begin{itemize}
		\item Ожидание пассивное, ресурсы CPU не тратятся.
		\item На каждое условие создаётся условная переменная.
		\item Поток, изменивший условие, может разбудить либо все ожидающие потоки (\t{broadcast}), либо один случайный (\t{signal}).
		\item Бывают spurious wakeup "--- система иногда может разбудить ждущий поток, даже если никто не вызывал \t{signal}/\t{broadcast}.
		\item Поэтому важно проверять условие после пробуждения (\t{while}, а не \t{if}).
	\end{itemize}
\end{frame}

\begin{frame}[fragile]{Создание}
	Точно так же, как и мьютекс:
\begin{minted}{cpp}
pthread_cond_t cond;
pthread_cond_init(&cond);
// ...
pthread_cond_destroy(&cond);
\end{minted}
\end{frame}

\begin{frame}[fragile]{Оповещение}
\begin{minted}{cpp}
pthread_mutex_t m;
pthread_cond_t cond; // GUARDED_BY(m)
bool some_condition; // GUARDED_BY(m)
// ...
pthread_mutex_lock(&m);
// Следующие две строки в любом порядке.
some_condition = true;
pthread_cond_signal(&cond);
pthread_mutex_unlock(&m);
\end{minted}
\end{frame}

\begin{frame}[fragile]{Ожидание условия}
\begin{minted}{cpp}
pthread_mutex_t m;
pthread_cond_t cond; // GUARDED_BY(m)
bool some_condition; // GUARDED_BY(m)
// ...
pthread_mutex_lock(&m);
while (!some_condition) {
    // Атомарно снимает мьютекс и начинает ожидание
    pthread_cond_wait(&cond, &m);
    // После выхода из cond_wait мьютекс снова захвачен.
}
pthread_mutex_unlock(&m);
\end{minted}
\end{frame}

\begin{frame}{Упражнение}
	\begin{enumerate}
		\item Возьмите реализацию с producer-consumer с \href{https://raw.githubusercontent.com/yeputons/spring-2019-paradigms/master/190410/sources/14-prod-cons.cpp}{GitHub}.
		\item Запустите и убедитесь, что на каждую введённую строчу отзывается второй поток: сначала сразу, а потом через две секунды.
		\item Убедитесь, что если во время ожидания второго потока ввести новую строчку, то на неё второй поток тоже среагирует.
		\item Убедитесь, что если во время ожидания ввести две новых строчки, то будет обработана только последняя.
		\item Задайте все вопросы по коду; поймите, зачем нужна и что делает каждая строчка.
		\item Есть ли проблемы в этом коде?
	\end{enumerate}
\end{frame}

\begin{frame}{Конечно, есть!}
	\begin{center}
		\includegraphics[scale=0.4]{multithreading-aliens.jpg}
	\end{center}
\end{frame}

\begin{frame}[t,fragile]{Проблема}
	Вот тут возникает race condition:
\begin{minted}{cpp}
while (true) {
    fgets(str, sizeof str, stdin);
    pthread_mutex_lock(&m);
    str_available = true;
    pthread_cond_signal(&cond);
    pthread_mutex_unlock(&m);
}
\end{minted}
	\pause
	\begin{itemize}
		\item \t{fgets} меняет буфер, который также читается из другого потока.
		\item Значит, буфер должен быть защищён мьютексом на всех стадиях.
		\item Если поменяем \t{fgets} и \t{pthread\_mutex\_lock} местами, то\only<1>{...}\only<2->{ будет deadlock: consumer не может читать данные, пока producer ждёт.}
		\only<3->{
		\item
			Правильно сначала считать в локальную переменную, а потом скопировать в буфер.
			\href{https://raw.githubusercontent.com/yeputons/spring-2019-paradigms/master/190410/sources/15-prod-cons-fixed.cpp}{Код}.
		}
	\end{itemize}
\end{frame}

\begin{frame}[t,fragile]{Проблема-2}
	Тут есть другая проблема (воспроизводится, например, под Linux):
\begin{minted}{cpp}
while (true) {
    pthread_mutex_lock(&m);
    fgets(str, sizeof str, stdin);
    str_available = true;
    pthread_cond_signal(&cond);
    pthread_mutex_unlock(&m);
}
\end{minted}
	\pause
	\begin{itemize}
		\item
			Может не гарантироваться <<честность>> мьютекcа.
		\item
			Первый поток его постоянно отпускает-захватывает, и хватает на долгое время.
			Второй поток никак не может его получить.
		\item
			Как решать?
		\only<3->{
		\item
			Читать из \t{stdin} в локальную переменную, потом копировать в общую память.
			Этакий канал.
		\item
			Мораль: не надо брать мьютекс сразу после отпускания.
		}
	\end{itemize}
\end{frame}

\begin{frame}{Резюме}
	\begin{itemize}
		\item Условные переменные нужны там и только там, где поток ждёт некоторое условие.
		\item А это условие всегда защищено ровно одним мьютексом (почему?)
		\item Соответственно, условная переменная тоже защищена ровно одним мьютексом.
		\item Условие всегда надо проверять в цикле.
		\item
			\t{pthread\_cond\_wait} "--- это лишь оптимизация.
			Если её заменить на отпускание/взятие мьютекса, программа должна остаться корректной.
		\item
			Никакого внутреннего состояния у условной переменной нет,
			из-за этого она просто реализуется в ОС, но программисту
			надо самому явно формулировать условие, которого ждёт поток.
		\item
			Условные переменные "--- не панацея, надо всё время думать: не берите мьютекс надолго.
	\end{itemize}
\end{frame}

\section{Заключение}

\begin{frame}{Заключение}
	\begin{itemize}
		\item Лучше пользоваться готовыми многопоточными библиотеками
		\item Но если захотелось свою "--- пишите без общей памяти (например, через каналы)
		\item Но если нужна общая память "--- пишите с мьютексами
		\item В домашке, скорее всего, потребуется реализовать многопоточную очередь (канал)
	\end{itemize}
\end{frame}


\end{document}
