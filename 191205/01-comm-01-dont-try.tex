\section{Обмен сообщениями}
\subsection{Не пытайтесь повторить это дома}

\begin{frame}
	\tableofcontents[currentsection,currentsubsection]
\end{frame}

\begin{frame}[fragile]{Загадка}
	Что произойдёт при запуске \href{https://raw.githubusercontent.com/yeputons/spring-2019-paradigms/master/190410/sources/12-optimizer.cpp}{кода}?
	Предполагаем, что запись и чтение \t{int} атомарны.
\begin{minted}{cpp}
int data;
void* worker(void*) {
    for (;;) {
        data++;
    }
}
// ...
    while (data < 100);
    printf("Done\n");
// ...
\end{minted}
	\begin{itemize}
		\pause\item Race condition отсутствуют.
		\pause\item Он зависнет.
		\pause\item И никогда не выведет \t{Done}.
	\end{itemize}
\end{frame}

\begin{frame}{Разгадка}
	\begin{center}
		\includegraphics[scale=0.3]{optimizer.jpg}
	\end{center}
	Как обычно в C/C++.
\end{frame}

\begin{frame}{Подробная разгадка}
	\begin{itemize}
		\item Компилятор по умолчанию ничего про потоки не знает.
		\item Очевидно, что \t{while (data < 100);} переменную \t{data} изменить не может.
		\item Соответственно, переменная \t{data} никак измениться не может.
		\item Значит, \t{data < 100} всегда истинно, можно заменить на \t{true}.
		\item Получаем бесконечный цикл.
			\begin{center}
				\includegraphics[scale=0.3]{win-lose.jpg}
			\end{center}
	\end{itemize}
\end{frame}

\begin{frame}[fragile]{volatile}
	Изменим \href{https://raw.githubusercontent.com/yeputons/spring-2019-paradigms/master/190410/sources/13-volatile.cpp}{код}:
\begin{minted}{cpp}
volatile int data;  // Обозначили переменную volatile.
void* worker(void*) {
    for (;;) {
        data++;
    }
}
// ...
    while (data < 100);
    printf("Done\n");
// ...
\end{minted}
	\begin{itemize}
		\item \t{volatile} говорит компилятору честно сохранять/читать значение этой переменной из памяти каждый раз, когда написано.
		\item Есть ли проблемы? \pause Пока нет.
	\end{itemize}
\end{frame}

\begin{frame}[fragile]{Reordering}
	Эквивалентны ли два куска кода?
	\begin{tabular}{p{0.4\textwidth}p{0.4\textwidth}}
		\centering
\begin{minted}{cpp}
int data, finished;
// ...
data = 123;
finished = 1;
\end{minted}
&
\begin{minted}{cpp}
int data, finished;
// ...
finished = 1;
data = 123;
\end{minted}
	\end{tabular}
	\pause
	\begin{itemize}
		\item Эквивалентны. Оптимизатор тоже так считает.
		\pause\item И может переставить местами: всё равно никто не заметит.
		\pause\item А что, если в другом потоке было так?
\begin{minted}{cpp}
if (finished) {
    printf("%d\n", data);
}
\end{minted}
	\end{itemize}
\end{frame}

\begin{frame}{Иллюстрация}
	\begin{center}
		\includegraphics[scale=0.2]{race-condition-knock-knock.jpg}
	\end{center}
\end{frame}

\begin{frame}{Reordeing возвращается}
	\begin{itemize}
		\item Даже если одна переменная помечена как \t{volatile}, компилятор может изменить порядок записи/чтения.
		\item А вот если обе "--- не может. Проблема решена? \pause
		\item В процессоре тоже есть оптимизатор.
		\item Он тоже может переставлять инструкции как захочет, а \t{volatile} действует только на компилятор.
		\item Есть специальные ассемблерные инструкции (<<барьеры памяти>>), которые действуют на процессор.
		\item Не надо сразу пытаться в этом разобраться.
		\item \t{volatile} не предназначен для многопоточности, он нужен для других целей (memory-mapped I/O).
		\item В любом случае, иногда один поток может встретить состояние, которое было бы невозможно получить, исполняя инструкции последовательно.
	\end{itemize}
\end{frame}

\begin{frame}{А что же mutex?}
	\begin{itemize}
		\item В разных языках/библиотеках разные модели памяти (потом должны подробно рассказать про Java).
		\item Обычно везде считается, что в следующих случаях происходит (почти) полная синхронизация памяти между двумя потоками:
			\begin{enumerate}
				\item $A$ взял мьютекс, который $B$ недавно отпустил (возможно, его брал ещё кто-то).
				\item $A$ создал поток $B$.
				\item $A$ подождал завершения потока $B$.
			\end{enumerate}
		\item Все нужные барьеры памяти и прочее уже вшиты внутрь мьютексов и работы с потоками.
	\end{itemize}
\end{frame}

\begin{frame}[fragile]{Простой пример}
\begin{minted}{cpp}
// Thread 1
started = true;
m.lock();
finished = true;
m.unlock();
// Thread 2
m.lock();
if (finished) {
    assert(started);    // Верно
}
m.unlock();
\end{minted}
	Если убрать из второго потока мьютекс "--- ничего не знаем.
\end{frame}

\begin{frame}[fragile]{Сложный пример}
\begin{minted}{cpp}
// Thread 1
started = true;
m.lock(); data++; m.unlock();
finished1 = true;
finished2 = true;
// Thread 2
m.lock(); m.unlock();
if (finished2) {
    assert(finished1);  // Непонятно, потому что reordering
    assert(data > 0);   // Непонятно, потому что нет(?) happens-before
    assert(started);    // Непонятно, потому что нет(?) happens-before
}
\end{minted}
\end{frame}

\begin{frame}[fragile]{Резюме}
	\begin{itemize}
		\item Если вы что-то не защитили мьютексом, можно огрести из-за reordering, даже если всё <<очевидно должно работать>>.
		\item Если всё защищено мьютексом и вы ничего не предполагаете о происходящем за пределами критических секций "--- не о чем беспокоиться.
		\item Ничего сложнее <<взяли один глобальный мьютекс перед операцией, отпустили в конце>> обычно не требуется (в том числе в дз).
		\item Любой сколько-нибудь более сложный контроль требует понимания модели памяти.
		\item
			Все проблемы "--- от общих ресурсов (переменные, файлы, экран).
			Поэтому стараются минимизировать их количество.
		\item Нет общих ресурсов "--- нет проблем.
	\end{itemize}
\end{frame}
